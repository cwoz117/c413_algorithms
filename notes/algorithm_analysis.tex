\documentclass{report}
\usepackage[hidelinks]{hyperref}
\usepackage{amsthm}
\usepackage{listings}

\theoremstyle{definition}
\newtheorem*{examp}{Ex}

\title{Algorithm Analysis Textbook Notes}
\author{Chris Wozniak}

\begin{document}
\maketitle
\tableofcontents
\chapter{Algorithm Analysis}
	Asymptotic analysis compares the resource consumption of a set of algorithms, comparing their runtime/memory use/disk space
	to determine which algorithm would be better to solve a particular solution. We can then rate their efficiencies as data sets become
	increasingly larger. The primary consideration when estimating an algorithms perfoamance is:
	\begin{itemize}
		\item \textbf{Basic Operations - } The number of operations required by the algorithm to process an input.
		\item \textbf{Input Size - } The number of inputs the algorithm must process to complete its task (How many
				Items a search algorithm must iterate through, number of elements to sort etc..)
	\end{itemize}
	Consider an algorithm to find the largest value in an array of $n$ integers. The algorithm looks at each integer, saving
	the largest value seen so far. It is often called the \textit{Largest-Value Sequential Search}
\begin{lstlisting}[language=c]
static int largest(int *a){
	int currrent_large = 0;
	for (int i = 1; i < sizeof(a); i++)
		if (a[current_large] < a[i])
			current_large = i;
	return a[current_large];
}
\end{lstlisting}
	We introduce a function $T(n) = cn$ to represent the growth rate for a running time of this algorithm. The value $c$ 
	represents all the constant time (incrementing $i$, comparing two values in the array etc..) and $n$ represents the 
	size of the array.
\newpage
	The \textbf{growth rate} for an algorithm is the rate at which the cost of the algorithm grows as the size of its input grows.
	There are several \textit{classes} of growth rates:
	\begin{itemize}
		\item \textbf{Linear - } $cn$, for some positive integer $c$ means that the running time grows proportionally to the 
			size of the input. ie: $10n, 2n$
		\item \textbf{Quadratic - } $cn^2$, $cn^3$ The highest order term is a square, cube, or some constant.
		\item \textbf{Exponential - } An exponential growth rate has the input size placed in the exponent: $2^n$ $n!$ is
			also a type of exponential growth.
	\end{itemize}
\section{Asymptotic Analysis}
	Asymptotic analysis is when we look primarily at the growth rate of an algorithm. Faster computers or shedding 
	some of the constant factors down could improve performace, but it would only change when the intersection would 
	occur, not whether it would occur. However, we must be aware that for small input sizes, sometimes higher performance
	algorithms aren't the best to use.
	\subsection{Bounds}
		We describe the estimations of running time in many ways, to help remove the ambiguity mentioned above. The
		\textbf{Upper Bound} running time represents the highest growth rate that an algorithm can have.
		\subsubsection{Upper Bound}
			If we were talking about a specific algorithm, which has the largest growth rate $f(n)$ in the worst case, we use Big-O
			notation: $O(f(n))$ \textit{in the worst case.} to describe our upper bound. (the most work it has to perform). It should
			be noted that we could also be talking about the upper bound in the best case, or average case as well.
		\subsubsection{Lower Bound}
			Similar to Upper bounds, but referring to the shortest growth rate seen by the algorithm, in best/average/worst use cases.
		\subsubsection{$\Theta$ notation}
			$\Theta$ notation is used when the Upper and Lower bounds are the same.
\newpage
	\subsection{Simplifying Rules}
		Once we determine the running-time equation for an algorithm it is simple to derive $O(n)$, $\Omega(n)$, $\Theta(n)$
		expressions. Use these following Rules:
		\begin{enumerate}
			\item If $f(n)$












\end{document}